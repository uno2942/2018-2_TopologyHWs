\documentclass{article} 
\usepackage{graphicx, amssymb}
\usepackage{amsmath}
\usepackage{amsfonts}
\usepackage{amsthm}
\usepackage{kotex}
\usepackage{bm}
\usepackage{hyperref}
\usepackage{xcolor}
\usepackage{mathrsfs}
\usepackage{tikz-cd}
\usepackage{mathtools}

\textwidth 6.5 truein 
\oddsidemargin 0 truein 
\evensidemargin -0.50 truein 
\topmargin -.5 truein 
\textheight 8.5in

\DeclareMathOperator{\cc}{\mathbb{C}}
\DeclareMathOperator{\rr}{\mathbb{R}}
\DeclareMathOperator{\bA}{\mathbb{A}}
\DeclareMathOperator{\fra}{\mathfrak{a}}
\DeclareMathOperator{\frb}{\mathfrak{b}}
\DeclareMathOperator{\frm}{\mathfrak{m}}
\DeclareMathOperator{\frp}{\mathfrak{p}}
\DeclareMathOperator{\Tr}{Tr}
\DeclareMathOperator{\slin}{\mathfrak{sl}}
\DeclareMathOperator{\Lie}{\mathsf{Lie}}
\DeclareMathOperator{\Alg}{\mathsf{Alg}}
\DeclareMathOperator{\Spec}{\mathrm{Spec}}
\DeclareMathOperator{\End}{\mathrm{End}}
\DeclareMathOperator{\rad}{\mathrm{rad}}
\DeclarePairedDelimiter\abs{\lvert}{\rvert}%
\newcommand{\id}{\mathrm{id}}
\newcommand{\Hom}{\mathrm{Hom}}
\newcommand{\Sch}{\mathbf{Sch}}
\newcommand{\Ring}{\mathbf{Ring}}
\newcommand{\T}{\mathcal{T}}
\newcommand{\B}{\mathcal{B}}
\newtheorem{lemma}{Lemma}
\newtheorem{theorem}{Theorem}


\begin{document}


\title{General Topology - HW 4}
\author{SungBin Park, Physics, 20150462} 

 \maketitle
\section*{Problem 8 in HW 3}
\section*{Problem 1}
Take arbitrary two point $x,y\in f(X)$. $f$ is surjection for codomain $f(X)$, so there exists $\alpha, \beta\in X$ such that $f(\alpha)=x$ and $f(\beta)=y$. Since $X$ is path connected, there exists continuous $\gamma:[a,b]\rightarrow X$, $a,b\in \rr$ such that $\gamma(a)=\alpha$, $\gamma(b)=\beta$. Consider $f\circ \gamma$, then $f\circ \gamma:[a,b]\rightarrow f(X)$ such that $f\circ \gamma(a)=f(\alpha)=x$ and $f\circ\gamma(b)=f(\beta)=y$. Then, it is path from $x$ to $y$, and $f(X)$ is path connected.
\section*{Problem 2}
Let $C$, $D$ be separation of $\bigcup A_n$, then for each $i$, $A_i\subset C$ or $D$ since $A_i$'s are connected. WLOG, let $A_1\subset C$, and $j$ be the smallest integer such that $A_j \subset D$ since $\mathbb{N}$ is well-ordered set. Then, $A_{j-1}\subset C$ and it means connected $A_{j-1}\cup A_{j}$ have a separation $C$, $D$... Therefore, $A_i\subset C$ for all $i$, and it is contradiction to existence of separation. Therefore, $\bigcup A_n$ is connected.
\section*{Problem 3}
Fix $x\in A_{\alpha_0}$ and let $\mathcal{C}_x=\{A_{\alpha}|x\in A_{\alpha}\}$, then $A_{\alpha_0}\bigcup_{A\in \mathcal{C}_x} A$ is connected. Let it $B_x$. Then, $\bigcup_{y\in A_{\alpha_0}} B_y$ is connected since $x\in A_{\alpha_0}\subset B_y$ for each $y\in A_{\alpha_0}$. Therefore, $\bigcup A_\alpha$ is connected.
\section*{Problem 4}
Let $\overline{A}\cap \overline{X-A}\cap C=\phi$. Then, $\left(\overline{A}\cap C\right)\cap\left(\overline{X-A}\cap C\right)=\phi$. However, $\left(\overline{A}\cap C\right)^c$ and $\left(\overline{X-A}\cap C\right)^c$ is open set in $C$ and forms separation of $C$ since $\left(\overline{A}\cap C\right)\cup \left(\overline{A}\cap C\right) = \left(\overline{A}\cup \overline{X-A}\right) \cap C=C$ and they are disjoint. Therefore, $C$ is not connected, which is contradiction. Therefore, $\overline{A}\cap \overline{X-A}\cap C\neq\phi$.
\section*{Problem 5}
I'll use modify the proof of the theorem in Munkers: \textit{A finite cartesian product of connected space is connected.}

Let's fix $a\in A^c$, $b\in B^c$ since they are proper subset. We know that $X\times b$(resp. $a\times Y$) is connected since it is homeomorphic wiht $X$(resp. $Y$). As a result, $(X\times b) \cup (a\times Y)$ is connected. Define
\begin{equation*}
\begin{split}
T_x&=(X\times b)\cup (x\times Y) \\
S_y&=(X\times y)\cup (a\times Y),
\end{split}
\end{equation*}
then $\bigcup_{x\in A^c} T_x$ and $\bigcup_{y\in B^c} S_y$ are connected since they shares common point $(a,b)$. By the same reason, $\left(\bigcup_{x\in A^c} T_x\right) \cup \left(\bigcup_{y\in B^c}\right)S_y$ is connected. They ...
\section*{Problem 6}
\begin{enumerate}
\item[A.] Let $K=\{\alpha_1, \ldots, \alpha_n\}$ and $\psi_K:X_K\rightarrow X^{n}$ such that $\psi_K=(\pi_{\alpha_1}, \ldots, \pi_{\alpha_n})$. By the definition of $X_K$, $\psi_K$ is bijective and bicontinuous. Thus $X_K$ is connected.
\item[B.] For all $K$, $a\in X_K$, so $\bigcup_{K\in F}X_K$ is connected.
\item[C.] Since $X\supset\overline{\bigcup_{K\in F}X_K}$, we need to show that $X\subset \overline{\bigcup_{K\in F}X_K}$. Let $\abs{K}<\infty$, then we can set $K=\{\alpha_1, \ldots, \alpha_{\abs{K}}\}$, so $X=\overline{\bigcup_{K\in F}X_K}$. Let $K$ is infinite set. Fix $x\in X$. For any open neighborhood of $x$ $U$, $\pi_{\alpha}(U)=X$ for all but finitely many. For the finite set $L$, let
\begin{equation*}
y=\begin{cases}
a_{\alpha} & \text{For }\alpha\notin L \\
x_{\alpha} & \text{For }\alpha\in L
\end{cases}
\end{equation*}
Then, $y\in U$. Therefore, $x\in \overline{\bigcup_{K\in F}X_K}$ and $X=\overline{\bigcup_{K\in F}X_K}$.
\end{enumerate}
\section*{Problem 7}
\section*{Problem 8}
\section*{Problem 9}
\section*{Problem 10}
\section*{Problem 11}
\section*{Problem 12}
\section*{Problem 13}
\section*{Problem 14}

\end{document}
