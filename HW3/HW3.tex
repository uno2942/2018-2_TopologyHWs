\documentclass{article} 
\usepackage{graphicx, amssymb}
\usepackage{amsmath}
\usepackage{amsfonts}
\usepackage{amsthm}
\usepackage{kotex}
\usepackage{bm}
\usepackage{hyperref}
\usepackage{xcolor}
\usepackage{mathrsfs}
\usepackage{tikz-cd}
\usepackage{mathtools}

\textwidth 6.5 truein 
\oddsidemargin 0 truein 
\evensidemargin -0.50 truein 
\topmargin -.5 truein 
\textheight 8.5in

\DeclareMathOperator{\cc}{\mathbb{C}}
\DeclareMathOperator{\rr}{\mathbb{R}}
\DeclareMathOperator{\bA}{\mathbb{A}}
\DeclareMathOperator{\fra}{\mathfrak{a}}
\DeclareMathOperator{\frb}{\mathfrak{b}}
\DeclareMathOperator{\frm}{\mathfrak{m}}
\DeclareMathOperator{\frp}{\mathfrak{p}}
\DeclareMathOperator{\Tr}{Tr}
\DeclareMathOperator{\slin}{\mathfrak{sl}}
\DeclareMathOperator{\Lie}{\mathsf{Lie}}
\DeclareMathOperator{\Alg}{\mathsf{Alg}}
\DeclareMathOperator{\Spec}{\mathrm{Spec}}
\DeclareMathOperator{\End}{\mathrm{End}}
\DeclareMathOperator{\rad}{\mathrm{rad}}
\DeclarePairedDelimiter\abs{\lvert}{\rvert}%
\newcommand{\id}{\mathrm{id}}
\newcommand{\Hom}{\mathrm{Hom}}
\newcommand{\Sch}{\mathbf{Sch}}
\newcommand{\Ring}{\mathbf{Ring}}
\newcommand{\T}{\mathcal{T}}
\newcommand{\B}{\mathcal{B}}
\newtheorem{lemma}{Lemma}
\newtheorem{theorem}{Theorem}


\begin{document}


\title{General Topology - HW 3}
\author{SungBin Park, Physics, 20150462} 

 \maketitle
\section*{Problem 1}
\begin{enumerate}
\item[A.] I'll first check the metric axioms.
\begin{enumerate}
\item $\bar{d}$ is a function
\begin{equation*}
\bar{d}:X\times X\longrightarrow \rr
\end{equation*}
and $\bar{d}(x,y)=\min\{d(x,y), 1\}\geq 0$
\item $\bar{d}(x,x)=d(x, x)=0$ for all $x$. Conversely, if $\bar{d}(x,y)=0$, then $\min\{d(x,y), 1\}=0\Rightarrow d(x,y)=0$ and $x=y$.
\item Symmetric: $\bar{d}(x,y)=\min\{d(x,y), 1\}=\min\{d(y,x), 1\}=\bar{d}(y,x)$.
\item $\bar{d}(x,y)+\bar{d}(y,z)=\min\{d(x,y), 1\}+\min\{d(y,z), 1\}= \min\{d(x,y)+d(y,z), d(x,y)+1, d(y,z)+1, 2\}\geq \min\{d(x,y)+d(y,z), 1\}\geq \min\{d(x,z), 1\}=\bar{d}(x,z)$
\end{enumerate}
Therefore, $\bar{d}$ is a metric.
\item[B.] I'll first check the metric axioms.
\begin{enumerate}
\item $\rho$ is a function
\begin{equation*}
\rho:X\times X\longrightarrow \rr
\end{equation*}
and $\rho(x,y)=\frac{d(x, y)}{d(x,y)+1}\geq 0$
\item $\rho(x,x)=\frac{d(x, x)}{d(x,x)+1}=0$ for all $x$. Conversely, if $\rho(x,y)=0$, then $d(x,y)=0$ and $x=y$.
\item Symmetric: $\rho(x,y)=\frac{d(x,y)}{d(x,y)+1}=\frac{d(y,x)}{d(y,x)+1}=\rho(y,x)$.
\item $\rho(x,y)+\rho(y,z)=\frac{d(x, y)}{d(x,y)+1}+\frac{d(y, z)}{d(y,z)+1}=2- \left(\frac{2+d(x, y)+d(y, z)}{d(x,y)d(y,z)+d(x,y)+d(y,z)+1}\right)\geq 2- \left(\frac{2+d(x, y)+d(y, z)}{d(x,y)+d(y,z)+1}\right)=2- \left(1+\frac{1}{d(x,y)+d(y,z)+1}\right)\geq 2- \left(1+\frac{1}{d(x,z)+1}\right)=\rho(x,z)$
\end{enumerate}
Therefore, $\rho$ is a metric. Since $f$ is bounded by $1$ for $t\geq 0$, $\rho$ is bounded metric.
\end{enumerate}
\section*{Problem 2}
Claim: $\overline{\rr^\infty}=\rr^\omega$ in $\rr^\omega$.
\begin{proof}
Let $x\in \rr^\omega$ and $U$ be an open neighborhood of $x$ in $\rr^\omega$. Since $p_i(U)=\rr$ for all $i$ but finitely many, let 
$i_{\max}$ be a natural number such that $p_i(U)= \rr$ for $i\geq i_{\max}$. Let $y\in R^\infty$ that $y_i=p_i(x)$ for $i<i_{\max}$ and 
$y=0$ for elsewhere. Then, $y\in U$. This is true for all open neighborhood of $x$, so $x\in \overline{\rr^\infty}$.
\end{proof}
\section*{Problem 3}
Claim: $\overline{\rr^\infty}=\rr^\infty \cup \{(x_i)\in \rr^\omega|\lim\limits_{i\rightarrow \infty}x_i\rightarrow 0\}$ in $\rr^\omega$.
\begin{proof}
Let $x\in \rr^\omega\setminus \rr^\infty$ and $U$ be an open neighborhood of $x$ in $\rr^\omega$. Since $x\notin \rr^\infty$, $x_i\neq 0$ 
for infinitely many $i$. Let the nonzero sequence $\{x_j\}$. Let $\lim\sup \abs{x_j}\neq 0$, then there exists subsequence of $\{\abs{x_j}\}$
 such that converges to $c\in (0,\infty]$ and $B(x, \min\{c/2, 1\})$ is a open neighborhood of $x$ disjoint with $\rr^\infty$.

Conversely, let $\lim\sup \abs{x_j}= 0$, then $\lim\limits_{j\rightarrow \infty} x_j= 0$ and for any $\epsilon>0$, $B(x, \epsilon)$ contains $0$ in $i$th coordinate for all $i$ but finitely many. Let $i_{\max}\in \mathbb{N}$ such that $0\in p_i(B(x, \epsilon))$ for $i\geq i_{\max}$. Then, $(y_i)\in \rr^\infty$ such that $y_i=x_i$ for $i< i_{\max}$ and $y=0$ for $i\geq i_{\max}$, so $(y_i)\in B(x,\epsilon)$. Therefore, $x\in \overline{R^\infty}$.

Consequently, $\overline{R^\infty}=\rr^\infty \cup \{(x_i)\in \rr^\omega|\lim\limits_{i\rightarrow \infty}x_i\rightarrow 0\}$.
\end{proof}
\section*{Problem 4}
First, I'll show that $D$ is a metric.
\begin{enumerate}
\item $\bar{d}$ is a function
\begin{equation*}
D:X\times X\longrightarrow \rr
\end{equation*}
and $D(x,y)\geq 0$
\item $D(x,x)=\sum\limits_{i=1}^n d(x_i, x_i)=0$ for all $x$. Conversely, if $D(x,y)=0$, then $ 0=D(x,y)\geq d(x_i, y_i)\geq 0$ for each $i$, so $x_i=y_i$ and $x=y$.
\item Symmetric: $D(x,y)=\sum\limits_{i=1}^n d(x_i, y_i)=\sum\limits_{i=1}^n d(y_i, x_i)=D(y,x)$.
\item $D(x,y)+D(y,z)=\sum\limits_{i=1}^n d(x_i, y_i)+ d(y_i, z_i)\geq \sum\limits_{i=1}^n d(x_i, z_i)=D(x,z)$
\end{enumerate}
Therefore, $\bar{D}$ is a metric.
Let $\B=\{B_1\times B_2\times\cdots\times B_n\subset \rr^n|B_i\text{ is an open interval of }\rr\}$ be a basis of $\rr^n$. Let $x\in B\in \B$, then there exists $\epsilon>0$ such that $B_d(x_i, \epsilon)\subset p_i(B)$ for all $i$ for each $p_i(B)$ is an open neighborhood and there exists small $\epsilon_i>0$ such that $B_d(x_i, \epsilon_i)\subset p_i(B)$ and we can set $\epsilon=\min\{\epsilon_i\}$. Therefore, $B_D(x, \epsilon)\subset B$ since for any $y\in B_D(x, \epsilon)$, $d(x_i, y_i)<\epsilon$.
\section*{Problem 5}
Let $d$ be a euclidean metric on $\rr$ and let $\bar{d}$ be a bounded metric on $\rr$ as in problem 1 (A). Let $\bar{d}_1$ be a function on $\rr^2$ such that
\begin{equation*}
\bar{d}_1(x,y)=
\begin{cases}
\bar{d}(x_2,y_2) & \text{if }x_1=y_1 \\
1 & \text{if } x_1\neq y_1.
\end{cases}
\end{equation*}
I'll show that $\bar{d}_1$ is a metric.
\begin{enumerate}
\item $\bar{d}_1(x,y)\geq 0$
\item $\bar{d}_1(x,x)=\bar{d}(x_2, x_2)=0$ for all $x$. Conversely, if $\bar{d}_1(x,y)=0$, then $x_1=y_1$ and $\bar{d}(x_2,y_2)=0$ implies $x_2=y_2$. Therefore, $x=y$.
\item Symmetric: If $x_1\neq y_1$, $\bar{d}_1(x,y)=1=\bar{d}_1(y,x)$. If $x_1=y_1$, $\bar{d}_1(x,y)=\bar{d}(x_2,y_2)=\bar{d}(y_2, x_2)=\bar{d}_1(y,x)$.
\item If $x_1\neq y_1$ or $y_1\neq z_1$, $\bar{d}_1(x,y)+\bar{d}_1(y,z)\leq 1 \leq \bar{d}_1(x,z)$. Conversely, if $x_1=y_1=z_1$, $\bar{d}_1(x,y)+\bar{d}_1(y,z)=\bar{d}(x_2,y_2)+\bar{d}(y_2,z_2)=\leq \bar{d}(x_2,z_2) = \bar{d}_1(x,z)$
\end{enumerate}
Therefore, $\bar{d}_1$ is a metric.

I need to show that $\bar{d}_1$ generate dictionary topology. In the last homework, we showed that the product topology on $\rr^d\times \rr$ is equal to the dictionary topology on $\rr^2$. Therefore, we can set the basis of dictionary topology on $\rr^2$ be $\B=\{\{a\}\times(b,c)|a,b,c\in\rr, -\infty<b<c<\infty\}$. For $x\in B=\{a\}\times(b,c)\in \B$, we can set $\epsilon=\min\{\frac{x_2-b}{2}, \frac{c-x_2}{2}\}$ so that $B_{\bar{d}_1}(x, \epsilon)\subset B$. Conversely, for any $y\in B_{\bar{d}_1}(x, \epsilon)$ for fixed $\epsilon$, we can set $\delta=\min\{\frac{\epsilon-(y_2-x_2)}{4}, \frac{\epsilon-(x_2-y_2)}{4}\}$ so that $\{y_1\}\times (y-\delta, y+\delta)\subset B_{\bar{d}_1}(x, \epsilon)$. (For $\epsilon\leq 1$, this can be viewed as setting small interval in the interval, and for $\epsilon>1$, we can arbitrary set $\delta>0$ since $B_{\bar{d}_1}(x, \epsilon)$ is the whole set.)

\section*{Problem 6}
Let $V$ be an open set in $Y$. I'll show that $f^{-1}(V)$ is open in $X$.

Let $x_0\in f^{-1}(V)$ and $y_0=f(x_0)\in V$. Since $V$ is an open set, there exists $\epsilon>0$ such that $B(y_0, \epsilon)\subset V$. Let $N$ be a natural number such that $\abs{f_n(x)-f(x)}\leq \epsilon/3$ for all $n\geq N$. Let $U=f^{-1}_N (B(y_0, \epsilon))$. Then $U$ is an open set containing $x_0$ since $\abs{f_N(x_0)-f(x_0)}=\abs{f_N(x_0)-y_0}\leq \epsilon$. If I show that $U$ is contained in $f^{-1}(V)$, then it implies $f$ is continuous function.

Let $x\in U$, I need to show that $f(x)\in V$ to show that $x\in f^{-1}(V)$, but $\abs{f(x)-y_0}\leq \abs{f(x)-f_N(x)}+\abs{f_N(x)-f_N(x_0)}+\abs{f_N(x_0)-f(x_0)}\leq \epsilon/3+\epsilon/3+\epsilon/3=\epsilon$. Therefore, $f(x)\in B(y_0,\epsilon)\subset V$ and $x\in f^{-1}(V)$.

Consequently, $f$ is a continuous function.
\section*{Problem 7}
($\Rightarrow$) Let a sequence $\{f_n\}$ of functions $f_n:Y\rightarrow\rr$ converges uniformly to a function $f:Y\rightarrow\rr$. Then, for any $\epsilon>0$, there exists $N(\epsilon)\in \mathbb{N}$ as a function of $\epsilon$ such that $\abs{f_n(y)-f(y)}< \epsilon$ for all $y\in Y$ for all $n\geq N$, so $\sup\{\abs{f_n(y)-f(y)}\}\leq \epsilon$ for all $n\geq N$ and it means $\rho(f_n-f)\leq \epsilon$ for $n\geq N$ in the metric space $(\rr^Y, \rho)$. Therefore, $\{f_n\}$ converges to $f$ in the metric space $(\rr^Y, \rho)$.

($\Leftarrow$) Let $\{f_n\}$ converges to $f$ in the metric space $(\rr^Y, \rho)$. Then for fixed $\epsilon>0$, there exists $N$ such that $\rho(f_n, f)=\sup\{\abs{f_n(y)-f(y)}\}\leq \epsilon/2$ for all $n\geq N$. It means $\abs{f_n(y)-f(y)}\leq \epsilon$ for all $y\in Y$ for all $n\geq N$. Therefore, the sequence $\{f_n\}$ converges uniformly to a function $f$.
\section*{Problem 8}
\begin{enumerate}
\item[A.] Since $p(\text{Im}q)=Y$, $p$ is surjective. Let $U$ be a open set in $X$. By the continuity of $q$, $q^{-1}(U)$ is open set in $Y$. Since $p\circ q=1_Y$, $p(U)=p\circ q(q^{-1}(U))=q^{-1}(U)$. Therefore $p$ is open map with continuity, and $p$ is quotient map.
\item[B.] Let the retraction $p$. It is definitely surjective and continuous. For a set $V\subset A$, let $p^{-1}(V)$ is open in $X$. Then $p^{-1}(V)\cap A=p_{A}^{-1}(V)=V$, and by subspace topology, $V$ is open in $A$. Therefore, $p$ is quotient topology.
\end{enumerate}
\section*{Problem 9}
\begin{enumerate}
\item[A.] I'll write the equivalence class of $\sim$ by $[(x,y)]=\{(z,w)\in \rr^2|z^2+w^2=x^2+y^2\}$. Let $p(z,w)=[(z,w)]$ and $\rr^2/\sim$ have quotient topology induced by $p$. Then, $p^{-1}[(z,1)]=g^{-1}(z)$ by the definition of $p$ and $g$. 

Let's define $f([x,y])=g(x,y)$ for the domain $\rr^2/\sim$. I need to show that this is a well-defined function. The $f$ definitely have function value for each element in $\rr^2/\sim$. Let $[(x,y)]=[(x',y')]$, then $x^2+y^2=x'^2+y'^2$ and $g(x,y)=g(x', y')$. Also, the codomain of $f$ is $[0,\infty)$. Therefore $f$ is well defined function. For $x^2+y^2\neq x'^2+y'^2$, $g(x,y)\neq g(x', y')$ so $f([(x,y)])\neq f([(x',y')])$. Also, for any $r\in [0, \infty)$, there exists $[(\sqrt{r}, 0)]$ such that $f([(\sqrt{r}, 0)])=r$. Therefore $f$ is bijective. For a basis $(a,b)$ or $[0, b)$, $a<b$, in $[0, \infty)$, considering the subspace topology as a subset of $\rr$, $g^{-1}(U)$ is open set in $\rr^2$.(Recall that we already showed such function is continuous in previous HW.) If we compute $p\circ g^{-1}(U)$, $p\circ g^{-1}(U)=\bigcup_{z\in U}p\circ g^{-1}(z)=\bigcup_{z\in U}[(z,1)]=\bigcup_{z\in U} f^{-1}(z)=f^{-1}(U)$. Also, $g^{-1}(z)=p^{-1}([(z,1)])$, $g^{-1}(U)=p^{-1}\left(\bigcup_{z\in U} [(z,1)]\right)$ and $p\circ g^{-1}(U)$ is open in $\rr^2$. Therefore, $f$ is continuous.

\item[B.] Define $h(r)=(r, 0)$ for $r\geq 0$. Then, $f(p(h(r)))=f(p(r, 0))=f([(r,0)])=r$. Therefore, $f\circ p \circ h=1_{[0,\infty)}$. Also, $h$ is continuous since for any basis $(a,b)\times (c,d)$, $a<b$, $c<d$ of $\rr^2$,
\begin{equation*}
h^{-1}\left((a,b)\times (c,d)\right)=\begin{cases}
\phi & \text{if }c\leq 0\leq d \\
(a,b) & \text{if } c<d<0 \text{ or }0<c<d.
\end{cases}
\end{equation*}
Therefore, $p\circ h$ is continuous left inverse of $f$, so $f$ is quotient map. Since $f$ is bijective, $\left(f^{-1}\right)^{-1}=f$. For any open set $U$ in $\rr^2/\sim$, there exists a set $V$ in $[0,\infty)$ such that $f^{-1}(V)=U$ and $V$ is open by the quotient map property of $f$. It implies $f(U)=V$, and $f$ is bicontinuous. Therefore, $f$ is homeomorphism.
\end{enumerate}
\section*{Problem 10}
Before starting, I'll prove a short lemma.
\begin{lemma}
$g:S^1\times S^1\rightarrow S^1$ by $g(z,w)=zw$ is continuous function.
\end{lemma}
\begin{proof}
First, I'll give a metric topology on $\cc$ with the usual topology $d(z,w)=\abs{z-w}$. Seeing $z=x+yi$, $w=s+ti$ and $\abs{z-w}=(x-s)^2+(t-y)^2$, the topological structure of $\cc$ is same as $\rr^2$ with euclidean metric. Therefore, I'll see $\cc$ as $\rr^2$ and $S^1=\{(x,y)\in \rr^2|x^2+y^2=1\}$. Let $h(x,y)=x^2+y^2$, then $S^1=h^{-1}(1)$, so $S^1$ is closed set in $\rr^2$.

Let $G:\rr^4\rightarrow \rr^2$ by $G(a,b,c,d)=(ac-bd, ad+bc)$, then $G$ is continuous map by the previous homework. Since $\rr^4$ with product topology has same topological structure with $\rr^2\times \rr^2$ with product topology, replacing the domain of $G$ by $\rr^2\times\rr^2$ does not change the continuity. By the restricting the domain of $\rr^2\times \rr^2$ by $S^1\times S^1$ with subspace topology, we can get a continuous function $g':S^1\times S^1\rightarrow \rr^2$ and since the codomain of $g'$ can be restrict to $S^1$, $g:S^1\times S^1\rightarrow S^1$ is continuous map. Since $g((a,b),(c,d))=(ac-bd, ad+bc)$ is the same as $g(z,w)=zw$ with $z=a+bi$, $w=c+di$, it does not depends on the defining space of $S^1$ whether it is $\rr^2$ or $\cc$. Therefore, $g$ is continuous.
\end{proof}

Let $p:S^1\times S^1\rightarrow X^*$ quotient map generating quotient topology on $X^*$ and define $g:S^1\times S^1\rightarrow S^1$ by $g(z,w)=zw$. Let $f$ be $f:X^*\rightarrow S^1$ by $f([(z,w)])=g(z,w)$. Then, $f$ have function value for each element in $X^*$ and $[(z,w)]=[(z',w')]\Rightarrow f([(z,w)])=f([(z',w')])$. Therefore, it is well-defined. Also it is bijective since $f(z,1)=z$ and $[(z,w)]\neq[(z',w')]\Rightarrow  f([(z,w)])\neq f([(z',w')])$.

Second, I'll prove that $g^{-1}(z)=p^{-1}([(z,1)])$ for $z\in S^1$. Let $\alpha\in g^{-1}(z)$ s,t, $\alpha\in S^1\times S^1$, then $g(\alpha)=z$ and $p(z)=[(z,1)]$. Conversely, if $\alpha\in p^{-1}([(z,1)])$, then $p(\alpha)=[(z,1)]$ and for $\alpha=(a,b)$, $ab=z$, so $\alpha\in g^{-1}(z)$. Therefore $g^{-1}(z)=p^{-1}([(z,1)])$.

Since $g$ is continuous by the lemma above, for any open set $U$ in $S^1$, then $g^{-1}(U)$ is open set in $S^1\times S^1$. Since $g^{-1}(U)=\bigcup_{z\in U} g^{-1}(z)$, $p\circ g^{-1}(U)=\bigcup_{z\in U} p\circ g^{-1}(z)=\bigcup_{z\in U} [(z,1)]=\bigcup_{z\in U}f^{-1}(z)=f^{-1}(U)$ and $g^{-1}(U)=\bigcup_{z\in U}g^{-1}(z)=\bigcup_{z\in U}p^{-1}([(z,1)])\Rightarrow p\circ g^{-1}(U)$ is open in $X^*$ and therefore, $f$ is continuous.

Let's define $h:S^1\rightarrow S^1\times S^1$ by $h(z)=(z, 1)$. Then, $f\circ p\circ h(z)=z$, so it is identity on $S^1$. Therefore, it is the same case as problem 9 B., and $f$ is hemeomorphism. Consequently, $X^*$ is hemeomorphic to $S^1$.
\end{document}
